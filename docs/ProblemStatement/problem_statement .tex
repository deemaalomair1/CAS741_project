\documentclass{article}

\usepackage{tabularx}
\usepackage{booktabs}
\usepackage{hyperref}
\usepackage[margin=3.8cm]{geometry}

\title{CAS 741: Problem Statement\\ Mass-Mass Stoichiometry Problem}

\author{Deemah Alomair , 400221796}

\date{}

\usepackage{color}

\newif\ifcomments\commentstrue

\ifcomments
\newcommand{\authornote}[3]{\textcolor{#1}{[#3 ---#2]}}
\newcommand{\todo}[1]{\textcolor{red}{[TODO: #1]}}
\else
\newcommand{\authornote}[3]{}
\newcommand{\todo}[1]{}
\fi

\newcommand{\wss}[1]{\authornote{blue}{SS}{#1}} 
\newcommand{\plt}[1]{\authornote{magenta}{TPLT}{#1}} %For explanation of the template
\newcommand{\an}[1]{\authornote{cyan}{Author}{#1}}


\begin{document}

\maketitle

\begin{table}[hp]
\caption{Revision History} \label{TblRevisionHistory}
\begin{tabularx}{\textwidth}{llX}
\toprule
\textbf{Date} & \textbf{Developer(s)} & \textbf{Change}\\
\midrule
16/09/2019 & Deemah Alomair & Describing the problem statement \\
18/09/2019 & Deemah Alomair & Revise the problem statement \\


... & ... & ...\\
\bottomrule
\end{tabularx}
\end{table}
\section*{Problem}
Chemical Stoichiometry is an important aspect of chemistry. It describes the relationship between chemical reactants and products in any chemical reaction. In real life, Chemical Stoichiometry is used for many purposes, and serves many real life applications. Like battery cells, in corrosion of House painting, amount of water pollution. For instance, one methane molecule and two oxygen molecules react to yield one carbon dioxide molecule and two water molecules. If we change the amount of one of these substances we will not get the desired result. However, one of most known problem in this field is to find a mass for any reactant. known as Mass-Mass Stoichiometry Problem.\footnote{\url{https://www.bucks.edu/media/bcccmedialibrary/tutoring/documents/chemistry/Stoichiometry.pdf}}
\section*{Proposed Solution}
The mass of any reactant is the main stone for many applications. Solving this problem can be done manually by any chemist, but  getting  the desired result (reactant mass) is not a single step. It needs to go through several steps. This consumes time, and effort especially if we have bunch of chemical reactions to solve. In addition, automatic program gives more accurate result than the manual method. Having a Stoichiometry Mass-Mass program that automate all needed steps to get the reactant mass is necessary. \\

Stoichiometry Mass-Mass program needs to convert unbalanced equation to balanced one as first stage by applying law of conservation of mass in chemistry. Then use the given mass of other reactant to get the amount of mole using mole-to-mass conversions. Using the known molar ratio of known reactant we can get the mole quantity for unknown reactant. Finally, use this mole to get the amount of reactant mass using the mole-to-mass conversions of that substance.
 \subsection*{Environment and Stakeholders}
 The environment of software is MacOS. The stakeholders of the project would be chemist or chemical students, Any one has an interest in chemistry, and application developers who need to get the mass of the reactant as a requirement of their application.
 



\end{document}
