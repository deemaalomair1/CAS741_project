\documentclass[12pt, titlepage]{article}

\usepackage{booktabs}
\usepackage{tabularx}
\usepackage{hyperref}
\hypersetup{
    colorlinks,
    citecolor=black,
    filecolor=black,
    linkcolor=red,
    urlcolor=blue
}
\usepackage[round]{natbib}

\usepackage{color}

\newif\ifcomments\commentstrue

\ifcomments
\newcommand{\authornote}[3]{\textcolor{#1}{[#3 ---#2]}}
\newcommand{\todo}[1]{\textcolor{red}{[TODO: #1]}}
\else
\newcommand{\authornote}[3]{}
\newcommand{\todo}[1]{}
\fi

\newcommand{\wss}[1]{\authornote{blue}{SS}{#1}} 
\newcommand{\plt}[1]{\authornote{magenta}{TPLT}{#1}} %For explanation of the template
\newcommand{\an}[1]{\authornote{cyan}{Author}{#1}}

%% Common Parts

\newcommand{\progname}{ProgName} % PUT YOUR PROGRAM NAME HERE %Every program
                                % should have a name


\begin{document}

\title{System Verification and Validation Plan for Stoichiometry Mass-Mass Program } 
\author{Deemah Alomair}
\date{\today}
	
\maketitle

\pagenumbering{roman}

\section{Revision History}

\begin{tabularx}{\textwidth}{p{3cm}p{2cm}X}
\toprule {\bf Date} & {\bf Version} & {\bf Notes}\\
\midrule
28/10/2019 & 1.0 & First version of document\\
\bottomrule
\end{tabularx}

\newpage

\tableofcontents

\newpage

\section{Symbols, Abbreviations and Acronyms}

\renewcommand{\arraystretch}{1.2}
\begin{tabular}{l l} 
  \toprule		
  \textbf{symbol} & \textbf{description}\\
  \midrule 
  T & Test\\
  SRS & Software Requirements Specification\\
  SMMP & Stoichiometry Mass-Mass Program \\
  R & Functional requirement\\
  NF & Non-functional requirement\\
  \bottomrule
\end{tabular}\\

\newpage

\pagenumbering{arabic}

\section{General Information}

This document is to build verification and validation plan for Stoichiometry
Mass-Mass Program and provide description about the testing that will be carried
out on the system. The main goal of this document is to check whether SMMP meets
the SRS and fulfill its intended purpose. This document will be used as the
reference and guidance for testing SMMP.

\subsection{Summary}

Stoichiometry Mass-Mass Program is a software that convert unbalanced chemical
reaction to balanced one. Then do some necessary calculations to get the mass of
one of the reactants involved in that chemical reaction.

\subsection{Objectives}


The objectives of the verification and validation plan is to ensure that SMMP is
Reliable. Building the confidence of the correctness of the software. In addition, 
enhance the maintainability for the traceability of the potential changes. Moreover,
ensure usability of the SMMP to novel users.

\subsection{Relevant Documentation}

Relevant Documents that need to be visited while reading this document include the following:
\newline
SRS, which can be found  \href{https://github.com/deemaalomair1/CAS741_project/tree/master/docs/SRS}{SRS} \cite{SoftwareSpecification}
\newline
System Design that includes both MIS and MG, which can be found \href{https://github.com/deemaalomair1/CAS741project/tree/master/docs/Design}{Design} \cite{Designdocument}
\newline
Unit Verification and Validation Plan,  which can be found \href{https://github.com/deemaalomair1/CAS741project/tree/master/docs/VnVPlan/UnitVnVPlan}{UnitVnVPlan} \cite{UnitVnVPlan}
\newline
System and Unit Verification and Validation Report ,  which can be found \href{https://github.com/deemaalomair1/CAS741project/tree/master/docs/VnVReport}{VnVReport} \cite{VnVReport}

\section{Plan}

\wss{There should generally be text between all section headings.  If there is
  no text, then you can provide a ``roadmap'' of the section.}

\subsection{Verification and Validation Team}

The test team has one member: Deemah Alomair

\wss{You should also list your classmates and the course instructor.}

\subsection{SRS Verification Plan}

SRS Verification Plan will include a feedback from the domain expert, secondary
reviewer and Dr.\ Smith.

\wss{I'm looking for more detail here and for the design verification.  Peter
  has a good start on integrating the plan with the course structure, including
  using our checklists.}

\subsection{Design Verification Plan}

Design Verification Plan will include a feedback from the domain expert,
secondary reviewer and Dr.\ Smith.

\subsection{Implementation Verification Plan}

Implementation Verification Plan will include the followings:
\begin{itemize}
\item Manual testing.
\item  Automatic testing that includes unit testing for each individual function and coverage testing using "unittest package" and "coverage package" that work with PyCharm environment.
\item  Pylint Linter will be considered throughout the implementation of SMMP.  
\item Parallel testing. 
\end{itemize}

\subsection{Software Validation Plan} N.A \wss{I agree, but you should explain
  to the reader why this is N.A.}

\section{System Test Description}

This section lists the system tests to be performed to verify whether or not the
program fulfills the functional requirements and to test how well it meets the
non-functional requirements. Note that some of the tests for the functional
requirements are unit tests.

\subsection{Tests for Functional Requirements}

\subsubsection{Input Tests}

\paragraph{Input constrains tests}

\begin{enumerate}

\item{\bf T1: Mass value test\\}

Control: Functional, dynamic, automatic.
					
Initial State: Not Applicable.
					
Input: -7. \wss{You should list all of the system level inputs.  Otherwise, the
  tester will not know what values to enter.  To prevent this document from
  getting too large, you can use a base case test with your innocuous inputs,
  and then use a table to show the delta from this base test case to give your
  erroneous input test cases.}
					
Output: Error message indicates that entered mass value should be greater than
0.

Test Case Derivation: The expected mass value is greater than 0.
					
How test will be performed: Automatic unit testing.

\item{\bf T2: Mass format test\\}

Control: Functional, dynamic, automatic.
					
Initial State: Not Applicable.
					
Input: ``H''.
					
Output: Error message indicates that entered mass value should be numeric.

Test Case Derivation: The expected entered mass value is a positive number.
					
How test will be performed: Automatic unit testing.

\item{\bf T3: Reactant/product format test\\}

Control: Functional, dynamic, automatic.
					
Initial State: Not Applicable.
					
Input: ``H2''.
					
Output: Error message indicates that entered Reactant value should be
combination of coefficient + chemical element. \wss{This error should be caught
  by a type error.}

Test Case Derivation: The expected entered reactant value is positive
coefficient + chemical element like : ``2H'' , ``$2H_2O$''
					
How test will be performed: Automatic unit testing.


\item{\bf T4: Chemical reaction format test\\}

Control: Functional, dynamic, automatic.
					
Initial State: Not Applicable.
					
Input: $"H_2O = 2"$ . \wss{Your inputs imply that you are planning on doing
  string parsing for your chemical reaction inputs.  I think this might be more
  work than you are prepared to do.  As we discussed, I think you can use a GUI
  with drop down menus and edit boxes to enter the parts of the equation.  You
  can avoid parsing that way.}
					
Output: Error message indicates that this is not a correct chemical reaction
format.

Test Case Derivation: The expected entered chemical reaction is set of reactants
and products like: $NaOH + H_2SO_4 \rightarrow H_2O + NA_2SO_4$
					
How test will be performed: Automatic unit testing.
 
\end{enumerate}

\subsubsection{Output Tests}

\paragraph{Output constrains tests}

\begin{enumerate}

\item{\bf T5: Result test\\}

Control: Functional, dynamic, automatic.
					
Initial State: Not Applicable.
					
Input: chemical reaction : "$NaOH + H_2SO_4 \rightarrow H_2O + NA_2SO_4$"\\ mass
of known reactant : ``3.10''\\ name of known reactant: "$H_2SO_4$"
				
Output: mass = 2.53 g.

Test Case Derivation: The expected value of the mass or error message otherwise.
					
How test will be performed: coverage testing, parallel testing. \wss{This is
  what kind of test it is, not how it will be performed.  Parallel with what?}

\item{\bf T6: Chemical reaction balancing test\\}

Type: Functional, dynamic, automatic.
					
Initial State: Not Applicable.
					
Input: $ N_2 + 3H_2 \rightarrow NH_3 $
					
Output: $ N_2 + 3H_2 \rightarrow 2NH_3 $ .

Test Case Derivation: The expected balance equation or error message otherwise.

How test will be performed: Parallel testing.

\end{enumerate}

\wss{You do not have nearly enough test cases with the actual functionality of
  your software.  You need multiple test cases to balance an equation and you
  need multiple test cases to determine the unknown masses.}

\subsection{Tests for Nonfunctional Requirements}

\subsubsection{Usability}		

\begin{enumerate}

\item{\bf T7: Usability testing\\}

Type: Nonfunctional,Dynamic, Manual.
					
Initial State: Not Applicable.
					
Input/Condition: Unbalanced chemical reaction/ mass of one reactant.
					
Output/Result: Second reactant mass.
					
How test will be performed: Ask domain expert to use the system then get a
feedback of the ease of using the system. \wss{You need more detail here.  This
  is not rigorous.  You should think from the perspective of a third party that
  you give this document and ask to complete the tests.  That third party needs
  enough information to complete the tests.}
					
\end{enumerate}

\subsubsection{Reliabilty \wss{spell check}}

\begin{enumerate}

\item{\bf T8: Reliability testing\\}

Type: Nonfunctional, Dynamic, Manual.
					
Initial State: Not Applicable.
					
Input/Condition: Unbalanced chemical reaction/ mass of one reactant.
					
Output/Result: Second reactant mass.
					
How test will be performed: using the system with different inputs and measuring
the percentage of correct answers. \wss{more detail - what inputs?, what
  outputs? how are the percentages calculated?}

\end{enumerate}

\subsubsection{Maintainability}

\begin{enumerate}

\item{\bf T9: Maintainability testing\\}

Type: Nonfunctional,Dynamic, Manual.
					
Initial State: Not Applicable.
					
Input/Condition: Existing SMMP system.
					
Output/Result: New version of SMMP.
					
How test will be performed: Try to add new feature or fix some constrains
\wss{proof read} and
observe system reaction.  \wss{How is this going to be measured?  What feature
  is being added?  Who is adding it?}

\end{enumerate}
\subsection{Traceability Between Test Cases and Requirements} A trace between
system tests and requirements is provided in
\hyperref[tab:reqtrace]{Table~\ref*{tab:reqtrace}}.

\begin{table}[h!]
\centering
\begin{tabular}{|c|c|c|c|c|c|c|c|c|c|}
\hline
	& T1 & T2 & T3 & T4 & T5 & T6 & T7 &T8  & T9 \\
\hline
R1  & X&X&X& X& & & & & \\ \hline
R2  & & & & & X& X& &  & \\ \hline
R3  & & & & &X & X& &   &\\ \hline
R4  &X & X& X& X&X & X& & & \\ \hline
NF1 & & & & & X& X& & X &  \\ \hline
NF2   & X& X& X&X & X& X& & & \\ \hline
NF3   & & & & & & &X & & \\ \hline
NF4   & & & & & & & & &X \\ \hline
\hline
\end{tabular}
\caption{Traceability Matrix Showing the Connections Between Requirements and system tests}
\label{tab:reqtrace}
\end{table}

\end{document}
