\documentclass[12pt, titlepage]{article}

\usepackage{amsmath, mathtools}

\usepackage[round]{natbib}
\usepackage{amsfonts}
\usepackage{amssymb}
\usepackage{graphicx}
\usepackage{colortbl}
\usepackage{xr}
\usepackage{hyperref}
\usepackage{longtable}
\usepackage{xfrac}
\usepackage{tabularx}
\usepackage{float}
\usepackage{siunitx}
\usepackage{booktabs}
\usepackage{multirow}
\usepackage[section]{placeins}
\usepackage{caption}
\usepackage{fullpage}

\hypersetup{
bookmarks=true,     % show bookmarks bar?
colorlinks=true,       % false: boxed links; true: colored links
linkcolor=red,          % color of internal links (change box color with linkbordercolor)
citecolor=blue,      % color of links to bibliography
filecolor=magenta,  % color of file links
urlcolor=cyan          % color of external links
}

\usepackage{array}

\externaldocument{../../SRS/SRS}

\usepackage{color}

\newif\ifcomments\commentstrue

\ifcomments
\newcommand{\authornote}[3]{\textcolor{#1}{[#3 ---#2]}}
\newcommand{\todo}[1]{\textcolor{red}{[TODO: #1]}}
\else
\newcommand{\authornote}[3]{}
\newcommand{\todo}[1]{}
\fi

\newcommand{\wss}[1]{\authornote{blue}{SS}{#1}} 
\newcommand{\plt}[1]{\authornote{magenta}{TPLT}{#1}} %For explanation of the template
\newcommand{\an}[1]{\authornote{cyan}{Author}{#1}}


\newcommand{\progname}{Program Name}

\begin{document}

\title{Module Interface Specification for Stoichiometry Mass-Mass Program}

\author{Deemah Alomair}

\date{\today}

\maketitle

\pagenumbering{roman}

\section{Revision History}

\begin{tabularx}{\textwidth}{p{3cm}p{2cm}X}
\toprule {\bf Date} & {\bf Version} & {\bf Notes}\\
\midrule
Date 1 & 1.0 & First version of document\\
\bottomrule
\end{tabularx}

~\newpage

\section{Symbols, Abbreviations and Acronyms}

See SRS Documentation  \href{https://github.com/deemaalomair1/CAS741_project/tree/master/docs/SRS}{here}


\newpage

\tableofcontents

\newpage

\pagenumbering{arabic}

\section{Introduction}

The following document details the Module Interface Specifications for
Stoichiometry Mass-Mass Program

Complementary documents include the System Requirement Specifications
and Module Guide.  The full documentation and implementation can be
found \href{https://github.com/deemaalomair1/CAS741_project/tree/master/docs}{here}

\section{Notation}

The structure of the MIS for modules comes from \cite{Hoffman:Strooper:1995},
with the addition that template modules have been adapted from
\cite{Ghezzi:Jazayeri:Mandrioli:2003}.  The mathematical notation comes from Chapter 3 of
\cite{Hoffman:Strooper:1995}.  For instance, the symbol := is used for a
multiple assignment statement and conditional rules follow the form $(c_1
\Rightarrow r_1 | c_2 \Rightarrow r_2 | ... | c_n \Rightarrow r_n )$.

The following table summarizes the primitive data types used by SMMP. 

\begin{center}
\renewcommand{\arraystretch}{1.2}
\noindent 
\begin{tabular}{l l p{7.5cm}} 
\toprule 
\textbf{Data Type} & \textbf{Notation} & \textbf{Description}\\ 
\midrule
character & char & a single symbol or digit\\
integer & $\mathbb{Z}$ & a number without a fractional component in (-$\infty$, $\infty$) \\
natural number & $\mathbb{N}$ & a number without a fractional component in [1, $\infty$) \\
real & $\mathbb{R}$ & any number in (-$\infty$, $\infty$)\\
\bottomrule
\end{tabular} 
\end{center}

\noindent
The specification of SMMP  uses some derived data types: sequences, strings, and
tuples. Sequences are lists filled with elements of the same data type. Strings
are sequences of characters. Tuples contain a list of values, potentially of
different types. In addition, SMMP  uses functions, which
are defined by the data types of their inputs and outputs. Local functions are
described by giving their type signature followed by their specification.

\noindent
Since chemical reaction is central to SMMP, the following notation
is introduced in an attempt to simplify the presentation of the MIS.\\

\noindent
Chemical reaction:\\

$c_1R[0]_1 + c_2R[0]_2 = cc_1R[1]_1 + cc_2R[1]_2$\\

where $R[0]_1$ is reactant with known mass, $c_1$ is the coefficient number associated to this reactant ,
$R[0]_2$  is reactant with unknown mass, $c_2$ is the coefficient number associated to this reactant , 
$R[1]_i$ is a product and  $cc_i$ is he coefficient number associated to this product.

\section{Module Decomposition}

The following table is taken directly from the Module Guide document for this project.

\begin{table}[h!]
\centering
\begin{tabular}{p{0.3\textwidth} p{0.6\textwidth}}
\toprule
\textbf{Level 1} & \textbf{Level 2}\\
\midrule

{Hardware-Hiding} & ~ \\
\midrule

\multirow{4}{0.3\textwidth}{Behaviour-Hiding}  &Input Module \\
& Atomic Mass Module\\
&Chemical Reaction Balancing Module\\
& Mass Calculation Module\\
\midrule

\multirow{1}{0.3\textwidth}{Software Decision} & GUI Module\\
\bottomrule

\end{tabular}
\caption{Module Hierarchy}
\label{TblMH}
\end{table}


~\newpage

\section{MIS of Input Module} \label{input} 

\subsection{Module}

Input

\subsection{Uses}

Not Applicable.

\subsection{Syntax}

\subsubsection{Exported Constants}
None.
\subsubsection{Exported Types}
ReactionT = ?
\subsubsection{Exported Access Programs}

\begin{center}
\begin{tabular}{p{2cm} p{4cm} p{4cm} p{2cm}}
\hline
\textbf{Name} & \textbf{In} & \textbf{Out} & \textbf{Exceptions} \\
\hline
{$input_1$} & chemical reaction: \newline ReactionT & - & -   \\
{$input_2$} & Mass: $\mathbb{R}$ & - & -   \\
\hline
\end{tabular}
\end{center}

\subsection{Semantics}

Not Applicable.

\newpage

\section{MIS of Atomic Mass Module} \label{Atomic Mass} 

\subsection{Module}

Atomic Mass

\subsection{Uses}

Balancing (Section \ref{balancing})

\subsection{Syntax}

\subsubsection{Exported Constants}

None. 
\subsubsection{Exported Types}

ElementT = ?


\subsubsection{Exported Access Programs}

\begin{center}
\begin{tabular}{p{2cm} p{4cm} p{4cm} p{2cm}}
\hline
\textbf{Name} & \textbf{In} & \textbf{Out} & \textbf{Exceptions} \\
\hline
{Atomic-Mass} & element: ElementT & Atomic-Mass: $\mathbb{R}$ & element $\notin$ ElementT  \\
\hline
\end{tabular}
\end{center}

\subsection{Semantics}

\subsubsection{State Variables}

1. e: ElementT
\newline 
2. Atomic-Mass: $\mathbb{R}$
\subsubsection{Environment Variables}

Atomic Mass library file.

\subsubsection{Assumptions}

None.

\subsubsection{Access Routine Semantics}

\noindent (Atomic Mass):
\begin{itemize}
\item transition: None.
\item output:  AtomicMass $\coloneqq$  $\mathbb{R}$
\item exception: if (e $\notin$ ElementT) $\rightarrow$  no-Atomic-Mass
\end{itemize}

\subsubsection{Local Functions}
AtomicMass: ElementT $\rightarrow$ $\mathbb{R}$ 
\newline
AtomicMass(e) $\equiv$ ( e= H $\rightarrow$ 1.0079 $\vert$  e= He $\rightarrow$ 4.002 $\vert$...)  

\newpage

\section{MIS of Chemical Reaction Balancing Module} \label{balancing} 

\subsection{Template Module}

Balancing

\subsection{Uses}

 Input (Section \ref{input})

\subsection{Syntax}

\subsubsection{Exported Constants}

None.

\subsubsection{Exported Types}

ElementT = $\forall$ e $\in$ set of chemical element = 
\{ H,He,Li,Be,B,C,N,O,F,Ne,Na,Mg,Al,Si,P,S,Cl,Ar,K,
Ca,Sc,Ti,V,Cr,Mn,Fe,Co,Ni,Cu,Zn,Ga,Ge,As,Se,Br,Kr,Rb,Sr,Y,Zr,Nb,Mo,Tc,Ru,Rh,Pd,Ag,Cd,In,
Sn,Sb,Te,I,Xe,Cs,Ba,La,Ce,Pr,Nd,Pm,Sm,Eu,Gd,Tb,Dy,Ho,Er,Tm,Yb,Lu,Hf,Ta,W,Re,Os,Ir,Pt,
Au,Hg,Tl,Pb,Bi,Po,At,Rn,Fr,Ra,Ac,Th,Pa,U,Np,Pu,Am,Cm,Bk,Cf,Es,Fm,Md,No,Lr,Rf,Db,Sg,
Bh,Hs,Mt,Ds,Rg,Cn,Nh,Fl,Mc,Lv,Ts,Og \} \\
\newline 
MoleculeT = tuple of (number :  $\mathbb{N}$ , element : ElementT).\\
\newline 
CompoundT = set of MoleculeT.\\
\newline 
StoichiometricT = tuple of (coefficient : N , Compound : CompoundT)\\
\newline
ChemicalEqT = set of StoichiometricT\\
\newline
ReactionT = sequence [2] of ChemicalEq.\\

\subsubsection{Exported Access Programs}

\begin{center}
\begin{tabular}{p{2cm} p{4cm} p{4cm} p{2cm}}
\hline
\textbf{Name} & \textbf{In} & \textbf{Out} & \textbf{Exceptions} \\
\hline
{Molecule's atoms} & Molecule: \newline MoleculeT &  Number-of-Atoms: $\mathbb{N}$  & -  \\
\hline
{compound's atoms} & Compound: \newline CompoundT &  Number-of-Atoms: $\mathbb{N}$  & -  \\
\hline
{Elements of a compound} & Compound: \newline CompoundT & set of Element: ElementT  & -  \\
\hline
{number of atoms in ChemicalEq} & Compound: \newline CompoundT
\newline coefficient :  $\mathbb{N}$ &  Number-of-Atoms: $\mathbb{N}$  & -  \\
\hline
{Elements of a ChemicalEq} & ChemicalEq: \newline ChemicalEqT & set of Element: ElementT  & -  \\
\hline
{Balanced reaction for an element} & Reaction: \newline ReactionT
 \newline Element: ElementT &  B: Boolean  & number of atoms in each side for one element is not equal \\
\hline
{Balanced reaction} & Reaction: \newline ReactionT &  B: Boolean  & number of atoms in each side for all element is not equal \\
\hline
{balanced chemical reaction} & Reaction: \newline ReactionT &  Reaction*: ReactionT  & number of atoms in each side for all element is not equal \\
\hline
\end{tabular}
\end{center}

\subsection{Semantics}

\subsubsection{State Variables}
e: ElementT
\newline
m : MoleculeT 
\newline
C: CompoundT
\newline
S: StoichiometricT
\newline
Ce: ChemicalEq
\newline
R: ReactionT
\subsubsection{Environment Variables}

None.

\subsubsection{Assumptions}

None.

\subsubsection{Access Routine Semantics}

\noindent (Number-of-Atoms-in-Molecule):
\begin{itemize}
\item transition: None.
\item output: Number-of-Atoms-in-Molecule $\coloneqq$ $\mathbb{N}$
\item exception:  None
\end{itemize}

\noindent (Number-of-Atoms-in-Compound):
\begin{itemize}
\item transition: None.
\item output: Number-of-Atoms-in-Compound $\coloneqq$ $\mathbb{N}$
\item exception:  None
\end{itemize}

\noindent (Number-of-Atoms-in-Stoichiometric):
\begin{itemize}
\item transition: None.
\item output: Number-of-Atoms-in-Stoichiometric $\coloneqq$  $\mathbb{N}$
\item exception:  None
\end{itemize}

\noindent (Number-of-Atoms-in-ChemicalEq):
\begin{itemize}
\item transition: None.
\item output: Number-of-Atoms-in-ChemicalEq $\coloneqq$  $\mathbb{N}$
\item exception:  None
\end{itemize}

\noindent (ElementsInCompoundT):
\begin{itemize}
\item transition: None.
\item output: set of ElementT 
\item exception:  None
\end{itemize}

\noindent (Elements of a ChemicalEq):
\begin{itemize}
\item transition: None.
\item output: set of ElementT
\item exception:  None
\end{itemize}

\noindent (Balanced reaction for an element):
\begin{itemize}
\item transition: None 
\item output: if (Number-of-Atoms-in-ChemicalEq(R[0],e) = Number-of-Atoms-in-ChemicalEq (R[1] ,e ) )
then True\\
if ( Number-of-Atoms-in-ChemicalEq(R[0],e) $\neq$ Number-of-Atoms-in-ChemicalEq (R[1] ,e ) ) then False
\item exception:  None
\end{itemize}

\noindent (Balanced reaction):
\begin{itemize}
\item transition: None.
\item output: if $\forall$( Number-of-Atoms-in-ChemicalEq(R[0],e) = Number-of-Atoms-in-ChemicalEq (R[1] ,e ) )
then True\\
if $\exists$( Number-of-Atoms-in-ChemicalEq(R[0],e) $\neq$ Number-of-Atoms-in-ChemicalEq (R[1] ,e ) )
then False
\item exception:  None
\end{itemize}


\subsubsection{Local Functions}
1. Number-of-Atoms-in-Molecule: MoleculeT  X ElementT $\rightarrow$ $\mathbb{N}$
\newline
Number-of-Atoms-in-Molecule(m,e) $\equiv$ (m.element = e $\rightarrow$ m.number $\vert$ m.element $\neq$ e $\rightarrow$ 0) \\
\newline
2. Number-of-Atoms-in-Compound: CompoundT  X ElementT $\rightarrow$ $\mathbb{N}$
\newline
Number-of-Atoms-in-Compound(C,e) $\equiv$ + (m : MoleculeT$\vert$ m $\in$ C $\cdot$   Number-of-Atoms-in-Molecule(m,e) ) \\
\newline
3. Number-of-Atoms-in-Stoichiometric: StoichiometricT  $\times$ ElementT $\rightarrow$ $\mathbb{N}$
\newline
Number-of-Atoms-in-Stoichiometric(S,e) $\equiv$ (S.coefficient $\cdot$ Number-of-Atoms-in-Compound (S.Compound,e) )  \\
\newline
4. Number-of-Atoms-in-ChemicalEq: ChemicalEqT  X ElementT $\rightarrow$ $\mathbb{N}$
\newline
Number-of-Atoms-in-ChemicalEq (Ce,e) $\equiv$ + (S : StoichiometricT $\vert$ S $\in$ Ce $\cdot$ Number-of-Atoms-in-Stoichiometric(S,e))\\
\newline
5. ElementsInCompoundT : CompoundT $\rightarrow$ set of ElementT.
\newline
ElementsInCompound(C) $\equiv$ $\cup$ \{ m : MoleculeT $\vert$ m $\in$ C $\cdot$ m.element\}\\
\newline
6. ElementsInChemicalEq : ChemicalEqT $\rightarrow$ set of ElementT.
\newline
ElementsInChemicalEq(Ce) $\equiv$ $\cup$ \{ C : CompoundT $\vert$ C $\in$ Ce $\cdot$ ElementsInCompound(C)\}\\
\newline
7. IsBalancedReactionForElement : ReactionT x ElementT $\rightarrow$ B 
\newline
IsBalancedReactionForElement (R, e) $\equiv$ Number-of-Atoms-in-ChemicalEq(R[0],e) = Number-of-Atoms-in-ChemicalEq (R[1] ,e )\\ 
\newline
8. IsBalancedReaction : ReactionT  $\rightarrow$ B 
\newline
IsBalancedReaction(R) $\equiv$ $\forall$ ( e :  ElementT $\vert$ e $\in$ ElementsInChemicalEq(R[0]) $\cdot$ IsBalancedReactionForElement(R, e))

\newpage
  
\section{MIS of Mass Calculation Module} \label{Mass} 

\subsection{Module}

Mass

\subsection{Uses}

Balancing (Section \ref{balancing}) , Atomic Mass (Section \ref{Atomic Mass})

\subsection{Syntax}

\subsubsection{Exported Constants}

None.

\subsubsection{Exported Access Programs}

\begin{center}
\begin{tabular}{p{2cm} p{4cm} p{4cm} p{2cm}}
\hline
\textbf{Name} & \textbf{In} & \textbf{Out} & \textbf{Exceptions} \\
\hline
{Molecular weight} & Atomic-Mass: $\mathbb{R}$
  \newline Number-of-Atoms-in-Compound: $\mathbb{N}$ & Molecular-weight:$\mathbb{R}$ & - \\
\hline
{mole-ratio}  & coefficient1 : $\mathbb{N}$
  \newline coefficient2 : $\mathbb{N}$ & mole-ratio: $\mathbb{N}$  & - \\
  \hline
{$mole_1$}  & mass : $\mathbb{R}$
  \newline Molecular-weight: $\mathbb{R}$ & mole: $\mathbb{R}$  & - \\
\hline
{$mole_2$} & mole1: $\mathbb{R}$
  \newline mole-ratio: $\mathbb{R}$ & mole2: $\mathbb{R}$ & - \\
 \hline
 {mass} & mole2: $\mathbb{R}$
  \newline Molecular-weight: $\mathbb{R}$ & mass:$\mathbb{R}$ & - \\
\hline
\end{tabular}
\end{center}

\subsection{Semantics}

\subsubsection{State Variables}

None


\subsubsection{Environment Variables}

None

\subsubsection{Assumptions}

None

\subsubsection{Access Routine Semantics}

\noindent (Molecular-weight):
\begin{itemize}
\item transition: None
\item output: Molecular-weight :=  $\mathbb{R}$ 
\item exception: None 
\end{itemize}

\noindent (Mass-to-Mole):
\begin{itemize}
\item transition: None
\item output: Mole := $\mathbb{R}$ 
\item exception: None 
\end{itemize}

\noindent (mole ratio):
\begin{itemize}
\item transition: None
\item output: mole-ratio := $\mathbb{N}$ 
\item exception: None 
\end{itemize}

\noindent (Mole-ratio-to-Mole):
\begin{itemize}
\item transition: None
\item output: Mole2 := $\mathbb{R}$
\item exception: None 
\end{itemize}

\noindent (Mole-to-Mass):
\begin{itemize}
\item transition: None
\item output:  Mass := $\mathbb{R}$
\item exception: None 
\end{itemize}

\subsubsection{Local Functions}
1. Molecular-weight : CompoundT  $\rightarrow$ $\mathbb{R}$ 
\newline
Molecular-weight := Atomic-Mass $\times$ Number-of-Atoms-in-Compound\\
\newline
2. Mole := $\frac{Mass}{Molecular_weight}$ \\
\newline
3. Mole-ratio:  CompoundT $\rightarrow$  $\mathbb{N}$
\newline
Mole-ratio := $\frac{coefficient2}{coefficient1}$\\
\newline
4. Mole2 := $\frac{Mole1}{mole-ratio}$ \\
\newline
5. Mass := $\frac{Mole}{Molecular-weight}$
\newpage

\section{MIS of GUI Module} \label{display} 

\subsection{Module}

GUI

\subsection{Uses}

Balancing  (Section \ref{balancing}) , Mass (Section \ref{Mass})


\subsection{Syntax}

\subsubsection{Exported Constants}

None.
\subsubsection{Exported Types}
ReactionT = ? 
\subsubsection{Exported Access Programs}

\begin{center}
\begin{tabular}{p{2cm} p{4cm} p{4cm} p{2cm}}
\hline
\textbf{Name} & \textbf{In} & \textbf{Out} & \textbf{Exceptions} \\
\hline
{GUI} & - & Mass: $\mathbb{R}$ \newline balanced-reaction: ReactionT & - \\
\hline
\end{tabular}
\end{center}

\subsection{Semantics}

\subsubsection{State Variables}

None

\subsubsection{Environment Variables}

screen interface

\subsubsection{Assumptions}

None

\subsubsection{Access Routine Semantics}

\noindent (display):
\begin{itemize}
\item transition: None
\item output: Mass: $\mathbb{R}$ , balanced reaction:  ReactionT 
\item exception: None
\end{itemize}


\subsubsection{Local Functions}

None


\bibliographystyle {plainnat}
\bibliography{../../ReferenceMaterial/References}
\newpage


\end{document}
